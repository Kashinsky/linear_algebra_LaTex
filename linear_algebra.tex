%%This is a very basic article template.
%%There is just one section and two subsections.
\documentclass{article}
\usepackage{amssymb,amsmath}
\begin{document}
\section{Problem 20}
\text{Let Q = (a,b,c,d,e). We want to solve for each coordinate such that:}\\
\begin{equation*}
\begin{split}
2(3,3,0,1,-1) + 3Q &= (20,17,1,10,-7)\\
(6,6,0,2,-2) + 3(a,b,c,d,e) &= (20,17,1,10,-7)\\
(6+3a,6+3b,3c,2+3d,-2+3e) &= (20,17,1,10,-7)\\
\end{split}
\end{equation*}
\text{Now just a matter of solving for each variable.}\\
\begin{equation*}
\begin{split}
6 + 3a &= 20\\
6 + 3b &= 17\\
0 + 3c &= 1\\
2 + 3d &= 10\\
-2 + 3e &= -7\\
\end{split}
\xrightarrow{}
\begin{split}
3a &= 14\\
3b &= 11\\
3c &= 1\\
3d &= 8\\
3e &= -5\\
\end{split}
\xrightarrow{}
\begin{split}
a &= \frac{14}{3}\\
b &= \frac{11}{3}\\
c &= \frac{1}{3}\\
d &= \frac{8}{3}\\
e &= \frac{-5}{3}\\
\end{split}
\end{equation*}
Finally we see,\\
\begin{equation*}
Q = \left(\frac{14}{3},\frac{11}{3},\frac{1}{3},\frac{8}{3},\frac{-5}{3}\right).
\end{equation*}

\newpage
\section{Problem 21}
For vectors (2,4,6,8,\ldots,40) and (1,3,5,7,\ldots,27), we want to find the
following information: length, the 13th entry and the formula for the kth
entry.
\subsection{Finding length}
At first glance, it is noticable that one vector consists of incrementing even
integers and the other consists of incrementing odd integers. Then we can find
the length using the definition of even and odd to find the index of the last
entries 40 and 27.
That is:\\
\begin{equation*}
\begin{split}
2k &= 40\\
&and\\
2j-1 &= 27
\end{split}
\xrightarrow{}
\begin{split}
k &= \frac{40}{2}\\
&and\\
j &= \frac{27+1}{2}
\end{split}
\xrightarrow{}
\begin{split}
k &= 20\\
&and\\
j &= 14
\end{split}
\end{equation*}
length of the first vector is 20 and the length of the second is 14
\subsection{Finding the 13th entry}
Using the same method I employed to find the length, I can find the 13th value
in both vectors.\\
Let i represent the index of each coordinate value in the first vector and let j
represent the index of each coordinate value in the second vector.\\
Let a and b represent the values of the coordinates at indeces i and j.\\
Then:\\
\begin{equation*}
\begin{split}
2i &= a\\
&and\\
2j-1 &= b\\
\end{split}
\xrightarrow{}
\begin{split}
2(13) &= a\\
and\\
2(13)-1 &= b\\
\end{split}
\xrightarrow{}
\begin{split}
26 &= a\\
&and\\
25 &= b\\
\end{split}
\end{equation*}
\subsection{Formula for the kth entry}
Since I accidentally found and utilized a method for finding entries in each
vector, I will use those to define functions for the value at that index such
that:\\
\begin{equation*}
\begin{split}
f(k)&=2k,\\
&and\\
g(k)&=2k-1
\end{split}
\end{equation*}
\newpage
\section{Problem 22}
\subsection{Compute}
\begin{equation*}
\begin{split}
(2, 3, -7, 1) &+ (8, 0, -1, -1)\\
(2+8, 3+0, -7&+(-1), 1+(-1))\\
(10,3,&-8,0)
\end{split}
\end{equation*}
And vice versa:\\
\begin{equation*}
\begin{split}
(8, 0, -1, -1) &+ (2, 3, -7, 1)\\
(8+2, 0+3, -1+&(-7), -1+(1))\\
(10, 3, &-8, 0)\\
\end{split}
\end{equation*}
\subsection{Adding generic vectors or length n}
To show that Q + P = P + Q, we will define generic vectors of arbitrary length
and perform the same operations in the last section.\\
Let $P = (a_1, a_2, a_3,\ldots,a_n)$ and $Q = (b_1, b_2, b_3, \ldots, b_n)$\\
\begin{equation*}
\begin{split}
(b_1, b_2, b_3, \ldots, b_n) + (a_1, a_2, a_3, \ldots, a_n) &= (a_1, a_2, a_3,
\ldots, a_n) + (b_1, b_2, b_3, \ldots, b_n)\\
(b_1 + a_1, b_2 + a_2, b_3 + a_3, \ldots, b_n + a_n) &= (a_1 + b_1, a_2 + b_2,
a_3 + b_3, \ldots, a_n + b_n)
\end{split}
\end{equation*}
Now by the our definition of vector equality,\\
\begin{equation*}
\begin{split}
b_1 + a_1 &= a_1 + b_1\\
b_2 + a_2 &= a_2 + b_2\\
b_3 + a_3 &= a_3 + b_3\\
\ldots &= \ldots\\
b_n + a_n &= a_n + b_n
\end{split}
\hspace{10pt}\text{By the commutative property of addition,}\hspace{10pt}
\begin{split}
b_1 + a_1 &= b_1 + a_1\\
b_2 + a_2 &= b_2 + a_2\\
b_3 + a_3 &= b_3 + a_3\\
\ldots &= \ldots\\
b_n + a_n &= b_n + a_n
\end{split}
\end{equation*}\\
Therefore, the commutative property is applicable to the addition of vectors of
equal length.
\newpage
\section{Problem 23}
To determine which of these vectors are in the span of $\mathbb{B} =
\{(3,3,0,1,-1),(-14,-11,-1,-8,5)\}$ we have to show it is a linear combination
of $\mathbb{B}$.
\subsection{(20, 17, 1, 10, -7)}
$\{t_1(3,3,0,1,-1) + t_2(-14,-11,-1,-8,5) = (20, 17, 1, 10, -7) \text{ where } 
t_1,t_2 \in \mathbb{R}\}$\\
\begin{equation*}
\begin{split}
t_1(3,3,0,1,-1) + t_2(-14,-11,-1,-8,5) &= (20, 17, 1, 10,
-7)\\
(3t_1,3t_1,0,t_1,-t_1) + (-14t_2,-11t_2,-1t_2,-8t_2,5t_2) &= (20, 17, 1,
10, -7)\\
(3t_1 + -14t_2, 3t_1 + -11t_2, 0 + -1t_2, t_1 + -8t_2, -t_1 + 5t_2) &= (20, 17, 1,
10, -7)
\end{split}
\end{equation*}
\begin{equation*}
\begin{split}
3t_1 + -14t_2 &= 20\\
3t_1 + -11t_2 &= 17\\
-1t_2 &= 1\\
t_1 + -8t_2 &= 10\\
-t_1 + 5t_2 &= -7
\end{split}
\end{equation*}
Solving for $t_2$ in the third equation, we can substitute its value
throughout the other equations.\\
\begin{equation*}
\begin{split}
3t_1 + -14t_2 &= 20\\
3t_1 + -11t_2 &= 17\\
t_2 &= -1\\
t_1 + -8t_2 &= 10\\
-t_1 + 5t_2 &= -7
\end{split}
\xrightarrow{}
\begin{split}
3t_1 + -14(-1) &= 20\\
3t_1 + -11(-1) &= 17\\
t_2 &= -1\\
t_1 + -8(-1) &= 10\\
-t_1 + 5(-1) &= -7
\end{split}
\xrightarrow{}
\begin{split}
3t_1 + 14 &= 20\\
3t_1 + 11 &= 17\\
t_2 &= -1\\
t_1 + 8 &= 10\\
-t_1 + -5 &= -7
\end{split}
\xrightarrow{}
\begin{split}
t_1 &= \frac{6}{3}\\
t_1 &= \frac{6}{3}\\
t_2 &= -1\\
t_1 &= 2\\
t_1 &= \frac{-2}{-1}
\end{split}
\xrightarrow{}
\begin{split}
t_1 &= 2\\
t_1 &= 2\\
t_2 &= -1\\
t_1 &= 2\\
t_1 &= 2
\end{split}
\end{equation*}\\
Since the system of equations does not lead to an error, we know $(20, 17, 1,
10, -7)$ is in the span of $\mathbb{B}$.\\
\newpage
\subsection{(0, 1, 4, 1, 1)}
$\{t_1(3,3,0,1,-1) + t_2(-14,-11,-1,-8,5) = (0, 1, 4, 1, 1) \text{ where } 
t_1,t_2 \in \mathbb{R}\}$\\
\begin{equation*}
\begin{split}
t_1(3,3,0,1,-1) + t_2(-14,-11,-1,-8,5) &= (0, 1, 4, 1, 1)\\
(3t_1,3t_1,0,t_1,-t_1) + (-14t_2,-11t_2,-1t_2,-8t_2,5t_2) &= (0, 1, 4, 1, 1)\\
(3t_1 + -14t_2, 3t_1 + -11t_2, 0 + -1t_2, t_1 + -8t_2, -t_1 + 5t_2) &= (0, 1, 4, 1, 1)
\end{split}
\end{equation*}
\\
\begin{equation*}
\begin{split}
3t_1 + -14t_2 &= 0\\
3t_1 + -11t_2 &= 1\\
-1t_2 &= 4\\
t_1 + -8t_2 &= 1\\
-t_1 + 5t_2 &= 1
\end{split}
\xrightarrow{}
\begin{split}
3t_1 + -14t_2 &= 0\\
3t_1 + -11t_2 &= 1\\
t_2 &= -4\\
t_1 + -8t_2 &= 1\\
-t_1 + 5t_2 &= 1
\end{split}
\xrightarrow{}
\begin{split}
3t_1 + -14(-4) &= 0\\
3t_1 + -11(-4) &= 1\\
t_2 &= -4\\
t_1 + -8(-4) &= 1\\
-t_1 + 5(-4) &= 1
\end{split}
\xrightarrow{}
\begin{split}
3t_1 + 56 &= 0\\
3t_1 + 56 &= 1\\
t_2 &= -4\\
t_1 + 32 &= 1\\
-t_1 + -20 &= 1
\end{split}
\xrightarrow{}
\begin{split}
t_1 &= \frac{-56}{3}\\
t_1 &= \frac{-55}{3}\\
t_2 &= -4\\
t_1 &= -31\\
t_1 &= -21
\end{split}
\end{equation*}\\
Since this system shows $t_1 = \frac{-56}{3} = -31 = -21$, and we know that it
is false, we can conclude that $(0, 1, 4, 1, 1)$ is not in the span of
$\mathbb{B}$.\\
\subsection{zero-vector}
$\{t_1(3,3,0,1,-1) + t_2(-14,-11,-1,-8,5) = (0, 0, 0, 0, 0) \text{ where } 
t_1,t_2 \in \mathbb{R}\}$\\
\begin{equation*}
\begin{split}
t_1(3,3,0,1,-1) + t_2(-14,-11,-1,-8,5) &= (0, 0, 0, 0, 0)\\
(3t_1,3t_1,0,t_1,-t_1) + (-14t_2,-11t_2,-1t_2,-8t_2,5t_2) &= (0, 0, 0, 0, 0)\\
(3t_1 + -14t_2, 3t_1 + -11t_2, 0 + -1t_2, t_1 + -8t_2, -t_1 + 5t_2) &= (0, 0, 0, 0, 0)
\end{split}
\end{equation*}
\\
\begin{equation*}
\begin{split}
3t_1 + -14t_2 &= 0\\
3t_1 + -11t_2 &= 0\\
-1t_2 &= 0\\
t_1 + -8t_2 &= 0\\
-t_1 + 5t_2 &= 0
\end{split}
\xrightarrow{}
\begin{split}
3t_1 + -14t_2 &= 0\\
3t_1 + -11t_2 &= 0\\
t_2 &= 0\\
t_1 + -8t_2 &= 0\\
-t_1 + 5t_2 &= 0
\end{split}
\xrightarrow{}
\begin{split}
3t_1 + 0 &= 0\\
3t_1 + 0 &= 0\\
t_2 &= 0\\
t_1 + 0 &= 0\\
-t_1 + 0 &= 0
\end{split}
\xrightarrow{}
\begin{split}
t_1 &= 0\\
t_1 &= 0\\
t_2 &= 0\\
t_1 &= 0\\
t_1 &= 0
\end{split}
\end{equation*}
Since the system of equations does not lead to an error, we know the zero vector
is in the span of $\mathbb{B}$.\\
%%\newpage
%%\section{Problem 24}

\end{document}
